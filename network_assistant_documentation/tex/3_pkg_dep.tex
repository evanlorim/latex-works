\section{Installing Package Dependencies}

\subsection{ncurses-devel}
ncurses-devel is needed for menuconfig
%would like to find which package has this
\begin{snugshade}\begin{verbatim}
$ sudo yum install ncurses-devel.x86_64 -y
\end{verbatim}\end{snugshade}\noindent

\hrule

\subsection{Python}
python-devel-2.6.6-37.el6.x86\_64 is needed for configuring the userland libraries.
\begin{snugshade}\begin{verbatim}
$ sudo yum install python-devel -y
\end{verbatim}\end{snugshade}\noindent

\hrule

\subsection{gtk2-devel}
gtk2-devel is needed for the gutil utility from the userland libraries
\begin{snugshade}\begin{verbatim}
$ sudo yum install gtk2-devel -y
\end{verbatim}\end{snugshade}\noindent
\hrule

\subsection{gcc Compatability Libraries}
Gcc compatability libraries are needed for the web100 libraries.
\begin{snugshade}\begin{lstlisting}
$ sudo yum install compat-gcc-34-g77
\end{lstlisting}\end{snugshade}\noindent


\subsection{Sun JDK 1.5 Dependencies}
The following 32-bit packages must be installed in order to get Sun JDK 1.5 working:
\begin{snugshade}\begin{lstlisting}
$ sudo yum install libXext.i686 libXt.i686 libXi.i686 libXp.i686 libXtst.i686 -y
\end{lstlisting}\end{snugshade}\noindent
The following 32-bit package needs to be installed to get JDK 1.5-alsa support:
\begin{snugshade}\begin{lstlisting}
$ sudo yum install alsa-lib.i686 -y
\end{lstlisting}\end{snugshade}\noindent
The following package needs to be installed to get JDK 1.5-fonts support:
\begin{snugshade}\begin{lstlisting}
$ sudo yum install chfontpath 
\end{lstlisting}\end{snugshade}\noindent
As well, we need to link our mkfntdir binary to the path that JDK-1.5-fonts searches:
\begin{snugshade}\begin{lstlisting}
$ sudo mkdir X11R6 X11R6/bin
$ sudo ln -s /usr/bin/mkfontdir /usr/X11R6/bin/mkfontdir
\end{lstlisting}\end{snugshade}\noindent

\hrule

\subsection{Sun JDK 1.5}
The following section is based off of instructions \href{http://wiki.centos.org/HowTos/JavaOnCentOS}{here} \footnote{http://wiki.centos.org/HowTos/JavaOnCentOS}.
\begin{itemize}
\item{The sun JDK 1.5 is needed in order to build ndt-3.6.4.}
\item{First, download the Sun JDK 1.5 update 15 rpm and put it in the home directory:}
\begin{snugshade}\begin{lstlisting}
$ cd ~
$ wget http://mirrors.dotsrc.org/jpackage/1.7/generic/SRPMS.non-free/java-1.5.0-sun-1.5.0.15-1jpp.nosrc.rpm
\end{lstlisting}\end{snugshade}\noindent
\item{Next download the Sun JDK 1.5 update 15 self extracting binary from here: \href{http://www.oracle.com/technetwork/java/archive-139210.html}{here}. \footnote{http://www.oracle.com/technetwork/java/archive-139210.html}.}
\item{It should be named \textit{jdk-1\_5\_0\_15-linux-i586.bin}}
\item{Move the downloaded file to the \textit{/home/seldon/rpmbuild/SOURCES} directory}
\item{Make it executable}
\end{itemize}
\begin{snugshade}\begin{lstlisting}
$ chmod +x ~/rpmbuild/SOURCES/jdk-1_5_0_15-linux-i586.bin
\end{lstlisting}\end{snugshade}\noindent
In order to build the package, you must set your computer to emulate a different architecture, that is: 
\begin{snugshade}\begin{lstlisting}
$ setarch i586
\end{lstlisting}\end{snugshade}\noindent
now build the rpm file:
\begin{snugshade}\begin{lstlisting}
$ rpmbuild --rebuild java-1.5.0-sun-1.5.0.15-1jpp.nosrc.rpm
\end{lstlisting}\end{snugshade}\noindent
if the installation goes over well you can \textit{unset} the i586 architecture:
\begin{snugshade}\begin{lstlisting}
$ exit
\end{lstlisting}\end{snugshade}\noindent
now install all of the rpms as root:
\begin{snugshade}\begin{lstlisting}
$ su 
# cd /home/seldon/rpmbuild/RPMS/i586
# rpm -Uvh java-1.5.0-sun-1.5.0.15-1jpp.i586.rpm
# rpm -Uvh java-1.5.0-sun-alsa-1.5.0.15-1jpp.i586.rpm
# rpm -Uvh --nodeps java-1.5.0-sun-fonts-1.5.0.15-1jpp.i586.rpm
# rpm -Uvh java-1.5.0-sun-plugin-1.5.0.15-1jpp.i586.rpm
# rpm -Uvh java-1.5.0-sun-devel-1.5.0.15-1jpp.i586.rpm
# exit
\end{lstlisting}\end{snugshade}\noindent










\newpage
