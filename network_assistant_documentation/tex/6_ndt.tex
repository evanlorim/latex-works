\section{The NDT Program}
This section follows from the NDT cookbook found \href{www.internet2.edu/pubs/ndt-cookbook.pdf‎}{here}.\footnote{www.internet2.edu/pubs/ndt-cookbook.pdf‎}
\subsection{Builing the NDT Program}
Make sure that you have installed the JDK version 1.5. To Configure the system to use JDK 1.5 do the following and choose jre 1.5: 
\begin{snugshade}\begin{lstlisting}
$ su
$ /usr/sbin/alternatives --config java
$ exit
\end{lstlisting}\end{snugshade}\noindent
Now download NDT:
\begin{snugshade}\begin{lstlisting}
$ cd /usr/src/etc
$ sudo wget http://software.internet2.edu/sources/ndt/ndt-3.6.2b.tar.gz
$ sudo tar -xvf ndt-3.6.2b.tar.gz
\end{lstlisting}\end{snugshade}\noindent
Now configure and make it:
\begin{snugshade}\begin{lstlisting}
$ cd /usr/src/etc/ndt-3.6.2b
$ sudo ./configure
$ sudo make
$ sudo make install
\end{lstlisting}\end{snugshade}\noindent

\hrule

\subsection{Testing the NDT program}
In the NDT source directory, run the shell script provided:
\begin{snugshade}\begin{lstlisting}
$ cd /usr/src/etc/ndt-3.6.2b 
$ sudo ./conf/create-html.sh
\end{lstlisting}\end{snugshade}\noindent
Following is how i filled out the prompt (where a blank means that I just pressed enter):
\begin{snugshade}\begin{lstlisting}
Enter your site name [Internet2]  : test_site_01
Enter your site's location [Ann Arbor - MI]  : St. Mary's - MD
Server connection info, enter 1 for 100 Mbps, 2 for 1 Gbps [2]  : 2
Enter email userid [rcarlson]  : hseldon
Enter email domain name [internet2.edu]  : smcm.edu
Enter default subject line [Trouble report from ndt3.smcm.edu]  : 
The base web page 'tcpbw100.html' has now been created.  You 
must move this file into the ndt_DATA directory [/usr/local/ndt]
created during the 'make' process.
Do you want to install this file now? [yes]  : 
Enter location [/usr/local/ndt]  : 
\end{lstlisting}\end{snugshade}\noindent
Start server processes with the provided sample script:
\begin{snugshade}\begin{lstlisting}
$ sudo ./conf/start.ndt
\end{lstlisting}\end{snugshade}\noindent
The first time I ran the script, I got the following output: 
\begin{snugshade}\begin{lstlisting}
[1] 3515
[2] 3520
\end{lstlisting}\end{snugshade}\noindent
copy the startup script to the init.d directory:
\begin{snugshade}\begin{lstlisting}
$ sudo cp ./conf/ndt /etc/init.d/ndt
\end{lstlisting}\end{snugshade}\noindent



