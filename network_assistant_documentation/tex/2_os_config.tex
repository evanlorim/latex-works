\section{Configuring CentOS}
\begin{itemize}
\item{Note that for all times that I user \textit{seldon} you can replace it with the user you registered.}
\item{Before doing anything else, install the software updates by going to the applications menu : administrative : software update and entering the root password (foobar)}
\end{itemize}

\subsection{Giving Root Priveleges}
\textit{Before doing the following, make sure you know basically how to use vim. Refer \href{http://maeks84.wordpress.com/2008/05/29/how-to-use-visudo/}{here} for how to use vim / visudo.}\footnote{http://maeks84.wordpress.com/2008/05/29/how-to-use-visudo/}\\\\
Open a terminal and type the following (password is foobar):
\begin{snugshade}\begin{verbatim}
	$ su
	# sudo /usr/bin/visudo
\end{verbatim}\end{snugshade}\noindent
Move to the bottom of the document, and add the following lines:
\begin{snugshade}\begin{verbatim}
#giving root priveleges to seldon
seldon	ALL=(ALL)	ALL
\end{verbatim}\end{snugshade}\noindent
Save and exit, and you should now be able to use sudo. Whenever you use sudo you will have to type in the users password, not the root password (foobar).\\
\hrule

\subsection{Enabling EPEL repository in yum}
\begin{snugshade}\begin{verbatim}
$ sudo rpm -Uvh http://mirrors.kernel.org/fedora-epel/6/i386/epel-release-6-8.noarch.rpm
\end{verbatim}\end{snugshade}\noindent
Check to ensure EPEL was enabled by issuing the command:
\begin{snugshade}\begin{verbatim}
$ yum repolist
\end{verbatim}\end{snugshade}\noindent
EPEL should be listed in there somewhere.\\
\hrule

\subsection{Enabling nux-desktop Repository in Yum}
\begin{snugshade}\begin{verbatim}
$ sudo rpm -Uvh http://li.nux.ro/download/nux/dextop/el6/x86_64/nux-dextop-release-0-1.el6.nux.noarch.rpm 
\end{verbatim}\end{snugshade}\noindent
Check to ensure nux-desktop was enabled by issuing the command:
\begin{snugshade}\begin{verbatim}
$ yum repolist
\end{verbatim}\end{snugshade}\noindent
nux-desktop should be listed somewhere in there
\hrule

\subsection{Installing Chromium (optional)}
Open a terminal and type the following:
\begin{snugshade}\begin{verbatim}
$ cd /etc/yum.repos.d
$ sudo wget http://people.centos.org/hughesjr/chromium/6/chromium-el6.repo
$ sudo yum install chromium -y
\end{verbatim}\end{snugshade}\noindent
You can use chromium with the chromium command or from the applications menu.\\
\hrule

\subsection{Installing Shutter (optional)}
Shutter is a tool for screen capture.
\begin{snugshade}\begin{verbatim}
$ sudo yum install shutter -y
\end{verbatim}\end{snugshade}\noindent
You can use shutter via the command shutter.\\
\hrule

\subsection{Setting Up RPM Build Environment}
These instructions are based off of the ones found \href{http://wiki.centos.org/HowTos/SetupRpmBuildEnvironment}{here}\footnote{http://wiki.centos.org/HowTos/SetupRpmBuildEnvironment}
\begin{snugshade}\begin{verbatim}
$ mkdir -p ~/rpmbuild/{BUILD,RPMS,SOURCES,SPECS,SRPMS}
$ echo '%_topdir %(echo $HOME)/rpmbuild' > ~/.rpmmacros
\end{verbatim}\end{snugshade}\noindent

\hrule

\newpage